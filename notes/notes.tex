\documentclass[12pt,a4paper]{book}

\usepackage[a4paper,text={16.5cm,25.2cm},centering]{geometry}
\usepackage{lmodern}
\usepackage{amssymb,amsmath}
\usepackage{bm}
\usepackage{graphicx}
\usepackage{microtype}
\usepackage{hyperref}
\usepackage{amsthm}
\usepackage{listings}
\setlength{\parindent}{0pt}
\setlength{\parskip}{1.2ex}
\let\QED=\blacksquare
\def\bbD{{\mathbb D}}
\def\bbZ{{\mathbb Z}}
\def\emdash{\hbox{---}}
\def\endash{\hbox{--}}
\def\nsubset{\not\subset}
\def\ldq{``}

\hypersetup
       {   pdfauthor = { {{Sheehan Olver}} },
           pdftitle={ {{MATH50003 Numerical Analysis}} },
           colorlinks=TRUE,
           linkcolor=black,
           citecolor=blue,
           urlcolor=blue
       }

\title{ MATH50003 Numerical Analysis }


\newtheorem{lemma}{Lemma}
\newtheorem{theorem}{Theorem}
\newtheorem{proposition}{Proposition}

\theoremstyle{definition}
\newtheorem{definition}{Definition}
\newtheorem{example}{Example}

\author{ Sheehan Olver }
\renewcommand{\thechapter}{\Roman{chapter}}


\def\addtab#1={#1\;&=}

\def\meeq#1{\def\ccr{\\\addtab}
%\tabskip=\@centering
 \begin{align*}
 \addtab#1
 \end{align*}
  }  
  
  \def\leqaddtab#1\leq{#1\;&\leq}
  \def\mleeq#1{\def\ccr{\\\addtab}
%\tabskip=\@centering
 \begin{align*}
 \leqaddtab#1
 \end{align*}
  }  


\def\vc#1{\mbox{\boldmath$#1$\unboldmath}}

\def\vcsmall#1{\mbox{\boldmath$\scriptstyle #1$\unboldmath}}

\def\vczero{{\mathbf 0}}


%\def\beginlist{\begin{itemize}}
%
%\def\endlist{\end{itemize}}


\def\pr(#1){\left({#1}\right)}
\def\br[#1]{\left[{#1}\right]}
\def\fbr[#1]{\!\left[{#1}\right]}
\def\set#1{\left\{{#1}\right\}}
\def\ip<#1>{\left\langle{#1}\right\rangle}
\def\iip<#1>{\left\langle\!\langle{#1}\right\rangle\!\rangle}

\def\norm#1{\left\| #1 \right\|}

\def\abs#1{\left|{#1}\right|}
\def\fpr(#1){\!\pr({#1})}

\def\Re{{\rm Re}\,}
\def\Im{{\rm Im}\,}

\def\floor#1{\left\lfloor#1\right\rfloor}
\def\ceil#1{\left\lceil#1\right\rceil}


\def\mapengine#1,#2.{\mapfunction{#1}\ifx\void#2\else\mapengine #2.\fi }

\def\map[#1]{\mapengine #1,\void.}

\def\mapenginesep_#1#2,#3.{\mapfunction{#2}\ifx\void#3\else#1\mapengine #3.\fi }

\def\mapsep_#1[#2]{\mapenginesep_{#1}#2,\void.}


\def\vcbr{\br}


\def\bvect[#1,#2]{
{
\def\dots{\cdots}
\def\mapfunction##1{\ | \  ##1}
\begin{pmatrix}
		 \,#1\map[#2]\,
\end{pmatrix}
}
}

\def\vect[#1]{
{\def\dots{\ldots}
	\vcbr[{#1}]
}}

\def\vectt[#1]{
{\def\dots{\ldots}
	\vect[{#1}]^{\top}
}}

\def\Vectt[#1]{
{
\def\mapfunction##1{##1 \cr} 
\def\dots{\vdots}
	\begin{bmatrix}
		\map[#1]
	\end{bmatrix}
}}



\def\thetaB{\mbox{\boldmath$\theta$}}
\def\zetaB{\mbox{\boldmath$\zeta$}}


\def\newterm#1{{\it #1}\index{#1}}


\def\TT{{\mathbb T}}
\def\C{{\mathbb C}}
\def\R{{\mathbb R}}
\def\II{{\mathbb I}}
\def\F{{\mathcal F}}
\def\E{{\rm e}}
\def\I{{\rm i}}
\def\D{{\rm d}}
\def\dx{\D x}
\def\ds{\D s}
\def\dt{\D t}
\def\CC{{\cal C}}
\def\DD{{\cal D}}
\def\U{{\mathbb U}}
\def\A{{\cal A}}
\def\K{{\cal K}}
\def\DTU{{\cal D}_{{\rm T} \rightarrow {\rm U}}}
\def\LL{{\cal L}}
\def\B{{\cal B}}
\def\T{{\cal T}}
\def\W{{\cal W}}


\def\tF_#1{{\tt F}_{#1}}
\def\Fm{\tF_m}
\def\Fab{\tF_{\alpha,\beta}}
\def\FC{\T}
\def\FCpmz{\FC^{\pm {\rm z}}}
\def\FCz{\FC^{\rm z}}

\def\tFC_#1{{\tt T}_{#1}}
\def\FCn{\tFC_n}

\def\rmz{{\rm z}}

\def\chapref#1{Chapter~\ref{Chapter:#1}}
\def\secref#1{Section~\ref{Section:#1}}
\def\exref#1{Exercise~\ref{Exercise:#1}}
\def\lmref#1{Lemma~\ref{Lemma:#1}}
\def\propref#1{Proposition~\ref{Proposition:#1}}
\def\warnref#1{Warning~\ref{Warning:#1}}
\def\thref#1{Theorem~\ref{Theorem:#1}}
\def\defref#1{Definition~\ref{Definition:#1}}
\def\probref#1{Problem~\ref{Problem:#1}}
\def\corref#1{Corollary~\ref{Corollary:#1}}

\def\sgn{{\rm sgn}\,}
\def\Ai{{\rm Ai}\,}
\def\Bi{{\rm Bi}\,}
\def\wind{{\rm wind}\,}
\def\erf{{\rm erf}\,}
\def\erfc{{\rm erfc}\,}
\def\qqquad{\qquad\quad}
\def\qqqquad{\qquad\qquad}


\def\spand{\hbox{ and }}
\def\spodd{\hbox{ odd}}
\def\speven{\hbox{ even}}
\def\qand{\quad\hbox{and}\quad}
\def\qqand{\qquad\hbox{and}\qquad}
\def\qfor{\quad\hbox{for}\quad}
\def\qqfor{\qquad\hbox{for}\qquad}
\def\qas{\quad\hbox{as}\quad}
\def\qqas{\qquad\hbox{as}\qquad}
\def\qor{\quad\hbox{or}\quad}
\def\qqor{\qquad\hbox{or}\qquad}
\def\qqwhere{\qquad\hbox{where}\qquad}



%%% Words

\def\naive{na\"\i ve\xspace}
\def\Jmap{Joukowsky map\xspace}
\def\Mobius{M\"obius\xspace}
\def\Holder{H\"older\xspace}
\def\Mathematica{{\sc Mathematica}\xspace}
\def\apriori{apriori\xspace}
\def\WHf{Weiner--Hopf factorization\xspace}
\def\WHfs{Weiner--Hopf factorizations\xspace}

\def\Jup{J_\uparrow^{-1}}
\def\Jdown{J_\downarrow^{-1}}
\def\Jin{J_+^{-1}}
\def\Jout{J_-^{-1}}



\def\bD{\D\!\!\!^-}




\def\questionequals{= \!\!\!\!\!\!{\scriptstyle ? \atop }\,\,\,}

\def\elll#1{\ell^{\lambda,#1}}
\def\elllp{\ell^{\lambda,p}}
\def\elllRp{\ell^{(\lambda,R),p}}


\def\elllRpz_#1{\ell_{#1{\rm z}}^{(\lambda,R),p}}


\def\sopmatrix#1{\begin{pmatrix}#1\end{pmatrix}}


\def\bbR{{\mathbb R}}
\def\bbC{{\mathbb C}}


\begin{document}

\maketitle

\tableofcontents

\chapter{Calculus on a Computer}

In this first chapter we explore the basics of mathematical computing and numerical analysis.
In particular we investigate the following mathematical problems which can not in general be solved exactly:

\begin{enumerate}
\item Integration. General integrals have no closed form expressions. Can we use a computer to approximate the values of definite integrals?
\item Differentiation. Differentiating a formula as in calculus is usually algorithmic, however, it is often needed to compute derivatives without access to an underlying formula, eg,  a function defined only in code. Can we use a computer to approximate derivatives?  A very important application is in Machine Learning, where there is a need to compute gradients to determine the ``right" weights in a neural network. 
\item Root finding. There is no general formula for finding roots (zeros) of arbitrary functions, or even polynomials that are of degree 5 (quintics) or higher. Can we compute roots of general functions using a computer?
\end{enumerate}

In this chapter we discuss:

\begin{enumerate}
\item I.1 Rectangular rule: we review the rectangular rule for integration and deduce the {\it converge rate} of the approximation. In the lab/problem sheet  we investigate its implementation as well as extensions to the Trapezium rule. 
\item I.2 Divided differences: we investigate approximating derivatives by a divided difference and again deduce the convergence rates. In the lab/problem sheet we extend the approach to the central differences formula and computing second derivatives. We also observe a mystery: the approximations may have significant errors in practice, and there is a limit to the accuracy.
\item I.3 Dual numbers: we introduce the algebraic notion of a {\it dual number} which allows the implemention of {\it forward-mode automatic differentiation}, a high accuracy alternative to divided differences for computing derivatives.
\item I.4 Newton's method: Newton's method is a basic approach for computing roots/zeros of a function. We use dual numbers to implement this algorithm.
\end{enumerate}




\section{Rectangular rule}
One possible definition for an integral is the limit of a Riemann sum, for example:
\[
  \ensuremath{\int}_a^b f(x) {\rm d}x = \lim_{n \ensuremath{\rightarrow} \ensuremath{\infty}} h \ensuremath{\sum}_{k=1}^n f(x_k)
\]
where $x_k = a+kh$ are evenly spaced points dividing up the interval $[a,b]$, that is  wit the \{{\textbackslash}it step size\} $h = (b-a)/n$. This suggests an algorithm known as the \emph{(right-sided) rectangular rule} for approximating an integral: choose $n$ large so that
\[
  \ensuremath{\int}_a^n f(x) {\rm d}x \ensuremath{\approx} h \ensuremath{\sum}_{k=1}^n f(x_k).
\]
In the lab we explore practical implementation of this approximation, and observe that the error in approximation is bounded by $C/n$ for some constant $C$. This can be expressed using "Big-O" notation:
\[
\ensuremath{\int}_a^b f(x) {\rm d}x = h \ensuremath{\sum}_{k=1}^n f(x_k) + O(1/n).
\]
In these notes we consider the "Analysis" part of "Numerical Analysis": we want to \emph{prove} the convergence rate of the approximation, including finding an explicit expression for the constant $C$.

To tackle this question we consider the error incurred on a single "rectangle", then sum up the errors on rectangles.

Now for a secret. There are only so many tools available in analysis (especially at this stage of your career), and  one can make a safe bet that the right tool in any analysis proof is either (1) integration-by-parts, (2) geometric series or (3) Taylor series. In this case we use (1):

\begin{lemma}[Rectangular Rule error on one panel] Assuming $f$ is differentiable we have
\[
\ensuremath{\int}_a^b f(x) {\rm d}x = (b-a) f(a) +  \ensuremath{\delta}
\]
where $|\ensuremath{\delta}| \ensuremath{\leq} M (b-a)^2$ for $M = \sup_{a \ensuremath{\leq} x \ensuremath{\leq} b}|f'(x)|$.

\end{lemma}
\textbf{Proof} We write
\meeq{
\ensuremath{\int}_a^b f(x) {\rm d}x = \ensuremath{\int}_a^b (x-a)' f(x)  {\rm d}x = [(x-a) f(x)]_a^b - \ensuremath{\int}_a^b (x-a) f'(x) {\rm d} x \ccr
= (b-a) f(b) + \underbrace{\left(-\ensuremath{\int}_a^b (x-a) f'(x) {\rm d} x \right)}_\ensuremath{\delta}.
}
Recall that we can bound the absolute value of an integral by the sepremum of the integrand times the width of the integration interval:
\[
\abs{\ensuremath{\int}_a^b g(x) {\rm d} x} \ensuremath{\leq} (b-a) \sup_{a \ensuremath{\leq} x \ensuremath{\leq} b}|g(x)|.
\]
The lemma thus follows since
\[
\abs{\ensuremath{\int}_a^b (x-a) f'(x) {\rm d} x} \ensuremath{\leq} (b-a) \sup_{a \ensuremath{\leq} x \ensuremath{\leq} b}|(x-a) f'(x)| \ensuremath{\leq} M (b-a)^2.
\]
\ensuremath{\QED}

Now summing up the errors in each panel gives us the error of using the Rectangular rule:

\begin{theorem}[Rectangular Rule error] Assuming $f$ is differentiable we have
\[
\ensuremath{\int}_a^b f(x) {\rm d}x =  h \ensuremath{\sum}_{k=1}^n f(x_k) +  \ensuremath{\delta}
\]
where $|\ensuremath{\delta}| \ensuremath{\leq} M (b-a) h$ for $M = \sup_{a \ensuremath{\leq} x \ensuremath{\leq} b}|f'(x)|$, $h = (b-a)/n$ and $x_k = a + kh$. 

\end{theorem}
\textbf{Proof} We split the integral into a sum of smaller integrals:
\[
\ensuremath{\int}_a^b f(x) {\rm d}x = \ensuremath{\sum}_{k=1}^n  \ensuremath{\int}_{x_{k-1}}^{x_k} f(x) {\rm d}x =
\ensuremath{\sum}_{k=1}^n  \br[(x_k - x_{k-1}) f(x_k) + \ensuremath{\delta}_k] =  h \ensuremath{\sum}_{k=1}^n f(x_k) +  \underbrace{\ensuremath{\sum}_{k=1}^n \ensuremath{\delta}_k}_\ensuremath{\delta}
\]
where $\ensuremath{\delta}_k$, the error on each panel as in the preceding lemma, satisfies 
\[
|\ensuremath{\delta}_k| \ensuremath{\leq} (x_k-x_{k-1})^2 \sup_{x_{k-1} \ensuremath{\leq} x \ensuremath{\leq} x_k}|f'(x)| \ensuremath{\leq} M h^2.
\]
Thus using the triangular inequality we have 
\[
|\ensuremath{\delta}| = \abs{ \ensuremath{\sum}_{k=1}^n \ensuremath{\delta}_k} \ensuremath{\leq} \ensuremath{\sum}_{k=1}^n |\ensuremath{\delta}_k| \ensuremath{\leq} M n h^2 = M(b-a)h.
\]
\ensuremath{\QED}

Note a consequence of this lemma is that the approximation converges as $n \ensuremath{\rightarrow} 0$. In the labs and problem sheets we will consider the left-sided rule:
\[
\ensuremath{\int}_a^b f(x) {\rm d}x \ensuremath{\approx}  h \ensuremath{\sum}_{k=0}^{n-1} f(x_k).
\]
We also consider the \emph{Trapezium rule}. Here we approximate an integral  by an affine function:
\[
\ensuremath{\int}_a^b f(x) {\rm d} x \ensuremath{\approx} \ensuremath{\int}_a^b {(b-x)f(a) + (x-a)f(b) \over b-a} \dx
= {b-a \over 2} \br[f(a) + f(b)].
\]
Subdividing an interval $a = x_0 < x_1 < \ensuremath{\ldots} < x_n = b$ and applying this approximation separately on each subinterval $[x_{k-1},x_k]$, where $h = (b-a)/n$ and $x_k = a + kh$, leads to the approximation
\[
\ensuremath{\int}_a^b f(x) {\rm d}x \ensuremath{\approx}  {h \over 2} f(a) + h \ensuremath{\sum}_{k=1}^{n-1} f(x_k) + {h \over 2} f(b)
\]
We shall see both experimentally and provably that this approximation converges faster than the rectangular rule.




\input{I.2.DividedDifferences.tex}

\section{Dual Numbers}
In this chapter we introduce a mathematically beautiful  alternative to divided differences for computing derivatives: \emph{dual numbers}. As we shall see, these are a commutative ring that \emph{exactly} compute derivatives, which when implemented in floating point give very high-accuracy approximations to derivatives. They underpin forward-mode \href{https://en.wikipedia.org/wiki/Automatic_differentiation}{automatic differentation}. Automatic differentiation  is a basic tool in Machine Learning for computing gradients necessary for training neural networks.

\begin{definition}[Dual numbers] Dual numbers $\ensuremath{\bbD}$ are a commutative ring (over $\ensuremath{\bbR}$) generated by $1$ and $\ensuremath{\epsilon}$ such that $\ensuremath{\epsilon}^2 = 0$. Dual numbers are typically written as $a + b \ensuremath{\epsilon}$ where $a$ and $b$ are real. \end{definition}

This is very much analoguous to complex numbers, which are a field generated by $1$ and $\I$ such that $\I^2 = -1$. Compare multiplication of each number type:
\meeq{
(a + b \I) (c + d \I) = ac + (bc + ad) \I + bd \I^2 = ac -bd + (bc + ad) \I \ccr
(a + b \ensuremath{\epsilon}) (c + d \ensuremath{\epsilon}) = ac + (bc + ad) \ensuremath{\epsilon} + bd \ensuremath{\epsilon}^2 = ac  + (bc + ad) \ensuremath{\epsilon} 
}
And just as we view $\ensuremath{\bbR} \ensuremath{\subset} \ensuremath{\bbC}$ by equating $a \ensuremath{\in} \ensuremath{\bbR}$ with $a + 0\I \ensuremath{\in} \ensuremath{\bbC}$, we can view $\ensuremath{\bbR} \ensuremath{\subset} \ensuremath{\bbD}$ by equating $a \ensuremath{\in} \ensuremath{\bbR}$ with $a + 0{\rm \ensuremath{\epsilon}} \ensuremath{\in} \ensuremath{\bbD}$.

\subsection{Differentiating polynomials}
Applying a polynomial to a dual number $a + b \ensuremath{\epsilon}$ tells us the derivative at $a$:

\begin{theorem}[polynomials on dual numbers] Suppose $p$ is a polynomial. Then
\[
p(a + b \ensuremath{\epsilon}) = p(a) + b p'(a) \ensuremath{\epsilon}
\]
\end{theorem}
\textbf{Proof}

It suffices to consider $p(x) = x^n$ for $n \ensuremath{\geq} 1$ as other polynomials follow from linearity. We proceed by induction: The case $n = 1$ is trivial. For $n > 1$ we have 
\[
(a + b \ensuremath{\epsilon})^n = (a + b \ensuremath{\epsilon}) (a + b \ensuremath{\epsilon})^{n-1} = (a + b \ensuremath{\epsilon}) (a^{n-1} + (n-1) b a^{n-2} \ensuremath{\epsilon}) = a^n + b n a^{n-1} \ensuremath{\epsilon}.
\]
\ensuremath{\QED}

\begin{example}[differentiating polynomial] Consider computing $p'(2)$ where
\[
p(x) = (x-1)(x-2) + x^2.
\]
We can use Dual numbers to differentiating, avoiding expanding in monomials or rules of differentiating:
\[
p(2+\ensuremath{\epsilon}) = (1+\ensuremath{\epsilon})\ensuremath{\epsilon} + (2+\ensuremath{\epsilon})^2 = \ensuremath{\epsilon} + 4 + 4\ensuremath{\epsilon} = 4 + \underbrace{5}_{p'(2)}\ensuremath{\epsilon}
\]
\end{example}

\subsection{Differentiating other functions}
We can extend real-valued differentiable functions to dual numbers in a similar manner. First, consider a standard function with a Taylor series (e.g. ${\rm cos}$, ${\rm sin}$, ${\rm exp}$, etc.)
\[
f(x) = \ensuremath{\sum}_{k=0}^\ensuremath{\infty} f_k x^k
\]
so that $a$ is inside the radius of convergence. This leads naturally to a definition on dual numbers:
\meeq{
f(a + b \ensuremath{\epsilon}) = \ensuremath{\sum}_{k=0}^\ensuremath{\infty} f_k (a + b \ensuremath{\epsilon})^k = f_0 + \ensuremath{\sum}_{k=1}^\ensuremath{\infty} f_k (a^k + k a^{k-1} b \ensuremath{\epsilon}) = \ensuremath{\sum}_{k=0}^\ensuremath{\infty} f_k a^k +  \ensuremath{\sum}_{k=1}^\ensuremath{\infty} f_k k a^{k-1} b \ensuremath{\epsilon}  \ccr
  = f(a) + b f'(a) \ensuremath{\epsilon}
}
More generally, given a differentiable function we can extend it to dual numbers:

\begin{definition}[dual extension] Suppose a real-valued function $f$ is differentiable at $a$. If
\[
f(a + b \ensuremath{\epsilon}) = f(a) + b f'(a) \ensuremath{\epsilon}
\]
then we say that it is a \emph{dual extension at} $a$.

Thus, for basic functions we have natural extensions:


\begin{align*}
\exp(a + b \ensuremath{\epsilon}) &:= \exp(a) + b \exp(a) \ensuremath{\epsilon} \\
\sin(a + b \ensuremath{\epsilon}) &:= \sin(a) + b \cos(a) \ensuremath{\epsilon} \\
\cos(a + b \ensuremath{\epsilon}) &:= \cos(a) - b \sin(a) \ensuremath{\epsilon} \\
\log(a + b \ensuremath{\epsilon}) &:= \log(a) + {b \over a} \ensuremath{\epsilon} \\
\sqrt{a+b \ensuremath{\epsilon}} &:= \sqrt{a} + {b \over 2 \sqrt{a}} \ensuremath{\epsilon} \\
|a + b \ensuremath{\epsilon}| &:= |a| + b\, {\rm sign} a\, \ensuremath{\epsilon}
\end{align*}
provided the function is differentiable at $a$. Note the last example does not have a convergent Taylor series (at 0) but we can still extend it where it is differentiable.

Going further, we can add, multiply, and compose such functions:

\begin{lemma}[product and chain rule] If $f$ is a dual extension at $g(a)$ and $g$ is a dual extension at $a$, then $q(x) := f(g(x))$ is a dual extension at $a$. If $f$ and $g$ are dual extensions at $a$ then  $r(x) := f(x) g(x)$ is also dual extensions at $a$. In other words:
\meeq{
q(a+b \ensuremath{\epsilon}) = q(a) + b q'(a) \ensuremath{\epsilon} \\
r(a+b \ensuremath{\epsilon}) = r(a) + b r'(a) \ensuremath{\epsilon}
}
\end{lemma}
\textbf{Proof} For $q$ it follows immediately:
\meeq{
q(a + b \ensuremath{\epsilon}) = f(g(a + b \ensuremath{\epsilon})) = f(g(a) + b g'(a) \ensuremath{\epsilon}) \ccr
= f(g(a)) + b g'(a) f'(g(a))\ensuremath{\epsilon} = q(a) + b q'(a) \ensuremath{\epsilon}.
}
For $r$ we have
\meeq{
r(a + b \ensuremath{\epsilon}) = f(a+b \ensuremath{\epsilon} )g(a+b \ensuremath{\epsilon} )= (f(a) + b f'(a) \ensuremath{\epsilon})(g(a) + b g'(a) \ensuremath{\epsilon}) \\
= f(a)g(a) + b (f'(a)g(a) + f(a)g'(a)) \ensuremath{\epsilon} = r(a) +b r'(a) \ensuremath{\epsilon}.
}
\end{definition}

A simple corollary is that any function defined in terms of addition, multiplication, composition, etc. of functions that are dual with differentiation will be differentiable via dual numbers.

\begin{example}[differentiating non-polynomial]

Consider $f(x) =  \exp(x^2 + \E^x)$ by evaluating on the duals:
\[
f(1 + \ensuremath{\epsilon}) = \exp(1 + 2\ensuremath{\epsilon} + \E + \E \ensuremath{\epsilon}) =  \exp(1 + \E) + \exp(1 + \E) (2 + \E) \ensuremath{\epsilon}
\]
and therefore we deduce that
\[
f'(1) = \exp(1 + \E) (2 + \E).
\]
\end{example}




\input{I.4.NewtonMethod.tex}


\chapter{Representing Numbers}

In this chapter we aim to answer the question: when can we rely on computations done on a computer?  Why are some computations (differentiation via divided differences), extremely inaccurate whilst others (integration via rectangular rule) accurate up to about 16 digits?  In order to address these questions we need to dig deeper and understand at a basic level what a computer is actually doing when manipulating numbers. 

Before we begin it is important to have a basic model of how a computer works. Our simplified model of a computer will consist of a \href{https://en.wikipedia.org/wiki/Central_processing_unit}{Central Processing Unit (CPU)}\ensuremath{\emdash}the  brains of the computer\ensuremath{\emdash}and \href{https://en.wikipedia.org/wiki/Computer_data_storage#Primary_storage}{Memory}\ensuremath{\emdash}where  data is stored. Inside the CPU there are \href{https://en.wikipedia.org/wiki/Processor_register}{registers}, where data is temporarily stored after being loaded from memory, manipulated by the CPU, then stored back to memory. 

Memory is a sequence of bits: \texttt{1}s and \texttt{0}s, essentially ``on/off" switches. These are grouped into bytes, which consist of 8 bits. Each byte has a memory address: a unique number specifying its location in memory. The number of possible addresses is limited by the processor: if a computer has a a $p$-bit CPU then each address is represented by $p$ bits, for a total of $2^p$ addresses (on a modern 64-bit CPU this is $2^{64} \ensuremath{\approx} 1.8 \times 10^{19}$ bytes). Further, each register consists of exactly $p$-bits.


Thus representing numbers on a computer must overcome three fundamental limitations:
\begin{enumerate}
\item CPUs can only manipulate data $p$-bits at a time.
\item Memory is finite, in particular at most $2^p$ bytes.
\item There is no such thing as an ``error'': if anything goes wrong in the computation we must use some of the $p$-bits to indicate this.
\end{enumerate}

This is clearly problematic: there are an infinite number of integers and an uncountable number of reals! Each of which we need to store in precisely $p$-bits. Moreover, some operations are simply undefined, like division by 0.  This chapter discusses the solution to this problem, alongside the mathematical analysis that is needed to understand the implications, in particular, that computations have {\it error}.

In particular we discuss:

\begin{enumerate}
\item II.1 Integers: unsigned (non-negative) and signed integers are representable using exactly $p$-bits by using modular arithmetic in all operations.
\item II.2 Reals:  real numbers are approximated by floating point numbers, which are the computers version of scientific notation.
\item II.3 Floating Point Arithmetic:  arithmetic with floating point numbers is exact up-to-rounding, which introduces small-but-understandable errors in the computations. We explain how these errors can be analysed mathematically to get rigorous bounds. 
\item II.4 Interval Arithmetic: rounding can be controlled in order to implement {\it interval arithmetic}, a way to compute rigorous bounds for computations. In the lab, we use this to compute up to 15 digits of ${\rm e} \equiv \exp 1$ rigorously with precise bounds on the error.
\end{enumerate}



\section{Integers}
In this section we discuss the following:

\begin{itemize}
\item[1. ] Binary representation: Any real number can be represented in binary, that is, by an infinite sequence of 0s and 1s (bits). We review  binary representation.


\item[2. ] Unsigned integers:  We discuss how computers represent non-negative integers using only $p$-bits, via \href{https://en.wikipedia.org/wiki/Modular_arithmetic}{modular arithmetic}.


\item[3. ] Signed integers: we discuss how negative integers are handled using the \href{https://en.wikipedia.org/wiki/Two's_complement}{Two's-complement} format.

\end{itemize}
Mathematically, CPUs only act on $p$-bits at a time, with $2^p$ possible sequences. That is, essentially all functions $f$ are either of the form $f : \ensuremath{\bbZ}_{2^p} \ensuremath{\rightarrow} \ensuremath{\bbZ}_{2^p}$ or  $f : \ensuremath{\bbZ}_{2^p} \ensuremath{\times} \ensuremath{\bbZ}_{2^p} \ensuremath{\rightarrow} \ensuremath{\bbZ}_{2^p}$, where we use the following notation:

\begin{definition}[signed integers] Denote the
\[
\ensuremath{\bbZ}_m := \{0 , 1 , \ensuremath{\ldots}, m-1 \}
\]
\end{definition}

The limitations this imposes on representing integers is substantial.  If we have an implementation of $+$, which we shall denote $\ensuremath{\oplus}_m$, how can we possibly represent $m + 1$ in this implementation when the result is above the largest possible integer?

The solution that is used is straightforward: the CPU uses modular arithmetic. E.g., we have
\[
(m-1) \ensuremath{\oplus}_m 1 = m\ ({\rm mod}\ m) = 0.
\]
In this chapter we discuss the implications of this approach and how it works with negative numbers.

We will write integers in binary format, that is, as sequence of \texttt{0}s and \texttt{1}s:

\begin{definition}[binary format] For $B_0,\ldots,B_p \in \{0,1\}$ denote an integer in \emph{binary format} by:
\[
\ensuremath{\pm}(B_p\ldots B_1B_0)_2 := \ensuremath{\pm}\sum_{k=0}^p B_k 2^k
\]
\end{definition}

\begin{example}[integers in binary] A simple integer example is $5 = 2^2 + 2^0 = (101)_2$. On the other hand, we write $-5 = -(101)_2$. Another example is $258 = 2^8 + 2 = (1000000010)_2$. \end{example}

\subsection{Unsigned Integers}
Computers represent integers by a finite number of $p$ bits, with $2^p$ possible combinations of 0s and 1s. For \emph{unsigned integers} (non-negative integers)  these bits dictate the first $p$ binary digits: $(B_{p-1}\ldots B_1B_0)_2$. 

Integers on a computer follow \href{https://en.wikipedia.org/wiki/Modular_arithmetic}{modular arithmetic}: Integers represented with $p$-bits on a computer actually  represent elements of ${\mathbb Z}_{2^p}$ and integer arithmetic on a computer is  equivalent to arithmetic modulo $2^p$. We denote modular arithmetic with $m = 2^p$ as follows:


\begin{align*}
x \ensuremath{\oplus}_m y &:= (x+y) ({\rm mod}\ m) \\
x \ensuremath{\ominus}_m y &:= (x-y) ({\rm mod}\ m) \\
x \ensuremath{\otimes}_m y &:= (x*y) ({\rm mod}\ m)
\end{align*}
When $m$ is implied by context we just write $\ensuremath{\oplus}, \ensuremath{\ominus}, \ensuremath{\otimes}$.

\begin{example}[arithmetic with  8-bit unsigned integers]  If  arithmetic lies between $0$ and $m = 2^8 = 256$ works as expected.  For example,


\begin{align*}
17 \ensuremath{\oplus}_{256} 3 = 20 ({\rm mod}\ 256) = 20 \\
17 \ensuremath{\ominus}_{256} 3 = 14 ({\rm mod}\ 256) = 14
\end{align*}
\end{example}

\begin{example}[overflow with 8-bit unsigned integers] If we go beyond the range the result ``wraps around". For example, with integers we have
\[
255 + 1 = (11111111)_2 + (00000001)_2 = (100000000)_2 = 256
\]
However, the result is impossible to store in just 8-bits!  So as mentioned instead it treats the integers as elements of ${\mathbb Z}_{256}$:
\[
255 \ensuremath{\oplus}_{256} 1 = 255 + 1 \ ({\rm mod}\ 256) = (00000000)_2 \ ({\rm mod}\ 256) = 0 \ ({\rm mod}\ 256)
\]
On the other hand, if we go below $0$ we wrap around from above:
\[
3 \ensuremath{\ominus}_{256} 5 = -2 ({\rm mod}\ 256) = 254 = (11111110)_2
\]
\end{example}

\begin{example}[multiplication of 8-bit unsigned integers]  Multiplication works similarly: for example,
\[
254 \ensuremath{\otimes}_{256} 2 = 254 * 2 \ ({\rm mod}\ 256) = 252 \ ({\rm mod}\ 256) = (11111100)_2 \ ({\rm mod}\ 256)
\]
\end{example}

\subsection{Signed integer}
Signed integers use the \href{https://epubs.siam.org/doi/abs/10.1137/1.9780898718072.ch3}{Two's complemement} convention. The convention is if the first bit is 1 then the number is negative: the number $2^p - y$ is interpreted as $-y$. Thus for $p = 8$ we are interpreting $2^7$ through $2^8-1$ as negative numbers. More precisely:

\textbf{Definition ($\ensuremath{\bbZ}_{2^p}^s$, unsigned integers)}
\[
\ensuremath{\bbZ}_{2^p}^s := \{-2^{p-1} ,\ensuremath{\ldots}, -1 ,0,1, \ensuremath{\ldots}, 2^{p-1}-1 \}
\]
\ensuremath{\QED}

\begin{definition}[Shifted mod] Define for $y = x\ ({\rm mod}\ 2^p)$
\[
x\ ({\rm mod}^{\rm s}\ 2^p) := \begin{cases} y & 0 \ensuremath{\leq} y \ensuremath{\leq} 2^{p-1}-1 \\
                             y - 2^p & 2^{p-1} \ensuremath{\leq} y \ensuremath{\leq} 2^p - 1
                             \end{cases}
\]
\end{definition}

Note that if $R_p(x) = x ({\rm mod}^{\rm s}\ 2^p)$ then it can be viewed as a map $R_p : \ensuremath{\bbZ} \ensuremath{\rightarrow} \ensuremath{\bbZ}_{2^p}^s$ or a one-to-one map $R_p : \ensuremath{\bbZ}_{2^p} \ensuremath{\rightarrow} \ensuremath{\bbZ}_{2^p}^s$ whose inverse is $R_p^{-1}(x) = x \mod 2^p$. 

Arithmetic works precisely the same for signed and unsigned integers, e.g. we have
\[
x \ensuremath{\oplus}_{2^p}^s y := x + y ({\rm mod}^{\rm s}\ 2^p)
\]
\begin{example}[addition of 8-bit integers] Consider \texttt{(-1) + 1} in 8-bit arithmetic. The number $-1$ has the same bits as $2^8 - 1 = 255$. Thus this is equivalent to the previous question and we get the correct result of \texttt{0}. In other words:
\[
-1 \ensuremath{\oplus}_{256} 1 = -1 + 1 \ ({\rm mod}\ 2^p) = 2^p-1  + 1 \ ({\rm mod}\ 2^p) = 2^p \ ({\rm mod}\ 2^p) = 0 \ ({\rm mod}\ 2^p)
\]
\end{example}

\begin{example}[multiplication of 8-bit integers] Consider \texttt{(-2) * 2}. $-2$ has the same bits as $2^{256} - 2 = 254$ and $-4$ has the same bits as $2^{256}-4 = 252$, and hence from the previous example we get the correct result of \texttt{-4}. In other words:
\[
(-2) \ensuremath{\otimes}_{2^p}^s 2 = (-2) * 2 \ ({\rm mod}^{\rm s}\ 2^p) = (2^p-2) * 2 \ ({\rm mod}^{\rm s}\ 2^p) = 2^{p+1}-4 \ ({\rm mod}^{\rm s}\ 2^p) = -4
\]
\end{example}




\input{II.2.Reals.tex}

\end{document}