\documentclass[12pt,a4paper]{article}

\usepackage[a4paper,text={16.5cm,25.2cm},centering]{geometry}
\usepackage{lmodern}
\usepackage{amssymb,amsmath}
\usepackage{bm}
\usepackage{graphicx}
\usepackage{microtype}
\usepackage{hyperref}
\usepackage[usenames,dvipsnames]{xcolor}
\setlength{\parindent}{0pt}
\setlength{\parskip}{1.2ex}




\hypersetup
       {   pdfauthor = {  },
           pdftitle={  },
           colorlinks=TRUE,
           linkcolor=black,
           citecolor=blue,
           urlcolor=blue
       }




\usepackage{upquote}
\usepackage{listings}
\usepackage{xcolor}
\lstset{
    basicstyle=\ttfamily\footnotesize,
    upquote=true,
    breaklines=true,
    breakindent=0pt,
    keepspaces=true,
    showspaces=false,
    columns=fullflexible,
    showtabs=false,
    showstringspaces=false,
    escapeinside={(*@}{@*)},
    extendedchars=true,
}
\newcommand{\HLJLt}[1]{#1}
\newcommand{\HLJLw}[1]{#1}
\newcommand{\HLJLe}[1]{#1}
\newcommand{\HLJLeB}[1]{#1}
\newcommand{\HLJLo}[1]{#1}
\newcommand{\HLJLk}[1]{\textcolor[RGB]{148,91,176}{\textbf{#1}}}
\newcommand{\HLJLkc}[1]{\textcolor[RGB]{59,151,46}{\textit{#1}}}
\newcommand{\HLJLkd}[1]{\textcolor[RGB]{214,102,97}{\textit{#1}}}
\newcommand{\HLJLkn}[1]{\textcolor[RGB]{148,91,176}{\textbf{#1}}}
\newcommand{\HLJLkp}[1]{\textcolor[RGB]{148,91,176}{\textbf{#1}}}
\newcommand{\HLJLkr}[1]{\textcolor[RGB]{148,91,176}{\textbf{#1}}}
\newcommand{\HLJLkt}[1]{\textcolor[RGB]{148,91,176}{\textbf{#1}}}
\newcommand{\HLJLn}[1]{#1}
\newcommand{\HLJLna}[1]{#1}
\newcommand{\HLJLnb}[1]{#1}
\newcommand{\HLJLnbp}[1]{#1}
\newcommand{\HLJLnc}[1]{#1}
\newcommand{\HLJLncB}[1]{#1}
\newcommand{\HLJLnd}[1]{\textcolor[RGB]{214,102,97}{#1}}
\newcommand{\HLJLne}[1]{#1}
\newcommand{\HLJLneB}[1]{#1}
\newcommand{\HLJLnf}[1]{\textcolor[RGB]{66,102,213}{#1}}
\newcommand{\HLJLnfm}[1]{\textcolor[RGB]{66,102,213}{#1}}
\newcommand{\HLJLnp}[1]{#1}
\newcommand{\HLJLnl}[1]{#1}
\newcommand{\HLJLnn}[1]{#1}
\newcommand{\HLJLno}[1]{#1}
\newcommand{\HLJLnt}[1]{#1}
\newcommand{\HLJLnv}[1]{#1}
\newcommand{\HLJLnvc}[1]{#1}
\newcommand{\HLJLnvg}[1]{#1}
\newcommand{\HLJLnvi}[1]{#1}
\newcommand{\HLJLnvm}[1]{#1}
\newcommand{\HLJLl}[1]{#1}
\newcommand{\HLJLld}[1]{\textcolor[RGB]{148,91,176}{\textit{#1}}}
\newcommand{\HLJLs}[1]{\textcolor[RGB]{201,61,57}{#1}}
\newcommand{\HLJLsa}[1]{\textcolor[RGB]{201,61,57}{#1}}
\newcommand{\HLJLsb}[1]{\textcolor[RGB]{201,61,57}{#1}}
\newcommand{\HLJLsc}[1]{\textcolor[RGB]{201,61,57}{#1}}
\newcommand{\HLJLsd}[1]{\textcolor[RGB]{201,61,57}{#1}}
\newcommand{\HLJLsdB}[1]{\textcolor[RGB]{201,61,57}{#1}}
\newcommand{\HLJLsdC}[1]{\textcolor[RGB]{201,61,57}{#1}}
\newcommand{\HLJLse}[1]{\textcolor[RGB]{59,151,46}{#1}}
\newcommand{\HLJLsh}[1]{\textcolor[RGB]{201,61,57}{#1}}
\newcommand{\HLJLsi}[1]{#1}
\newcommand{\HLJLso}[1]{\textcolor[RGB]{201,61,57}{#1}}
\newcommand{\HLJLsr}[1]{\textcolor[RGB]{201,61,57}{#1}}
\newcommand{\HLJLss}[1]{\textcolor[RGB]{201,61,57}{#1}}
\newcommand{\HLJLssB}[1]{\textcolor[RGB]{201,61,57}{#1}}
\newcommand{\HLJLnB}[1]{\textcolor[RGB]{59,151,46}{#1}}
\newcommand{\HLJLnbB}[1]{\textcolor[RGB]{59,151,46}{#1}}
\newcommand{\HLJLnfB}[1]{\textcolor[RGB]{59,151,46}{#1}}
\newcommand{\HLJLnh}[1]{\textcolor[RGB]{59,151,46}{#1}}
\newcommand{\HLJLni}[1]{\textcolor[RGB]{59,151,46}{#1}}
\newcommand{\HLJLnil}[1]{\textcolor[RGB]{59,151,46}{#1}}
\newcommand{\HLJLnoB}[1]{\textcolor[RGB]{59,151,46}{#1}}
\newcommand{\HLJLoB}[1]{\textcolor[RGB]{102,102,102}{\textbf{#1}}}
\newcommand{\HLJLow}[1]{\textcolor[RGB]{102,102,102}{\textbf{#1}}}
\newcommand{\HLJLp}[1]{#1}
\newcommand{\HLJLc}[1]{\textcolor[RGB]{153,153,119}{\textit{#1}}}
\newcommand{\HLJLch}[1]{\textcolor[RGB]{153,153,119}{\textit{#1}}}
\newcommand{\HLJLcm}[1]{\textcolor[RGB]{153,153,119}{\textit{#1}}}
\newcommand{\HLJLcp}[1]{\textcolor[RGB]{153,153,119}{\textit{#1}}}
\newcommand{\HLJLcpB}[1]{\textcolor[RGB]{153,153,119}{\textit{#1}}}
\newcommand{\HLJLcs}[1]{\textcolor[RGB]{153,153,119}{\textit{#1}}}
\newcommand{\HLJLcsB}[1]{\textcolor[RGB]{153,153,119}{\textit{#1}}}
\newcommand{\HLJLg}[1]{#1}
\newcommand{\HLJLgd}[1]{#1}
\newcommand{\HLJLge}[1]{#1}
\newcommand{\HLJLgeB}[1]{#1}
\newcommand{\HLJLgh}[1]{#1}
\newcommand{\HLJLgi}[1]{#1}
\newcommand{\HLJLgo}[1]{#1}
\newcommand{\HLJLgp}[1]{#1}
\newcommand{\HLJLgs}[1]{#1}
\newcommand{\HLJLgsB}[1]{#1}
\newcommand{\HLJLgt}[1]{#1}


\def\endash{–}
\def\bbD{ {\mathbb D} }
\def\bbZ{ {\mathbb Z} }

\def\x{ {\vc x} }
\def\a{ {\vc a} }
\def\b{ {\vc b} }
\def\e{ {\vc e} }
\def\f{ {\vc f} }
\def\u{ {\vc u} }

\def\red#1{ {\color{red} #1} }
\def\blue#1{ {\color{blue} #1} }
\def\green#1{ {\color{ForestGreen} #1} }
\def\magenta#1{ {\color{magenta} #1} }


\def\addtab#1={#1\;&=}

\def\meeq#1{\def\ccr{\\\addtab}
%\tabskip=\@centering
 \begin{align*}
 \addtab#1
 \end{align*}
  }  
  
  \def\leqaddtab#1\leq{#1\;&\leq}
  \def\mleeq#1{\def\ccr{\\\addtab}
%\tabskip=\@centering
 \begin{align*}
 \leqaddtab#1
 \end{align*}
  }  


\def\vc#1{\mbox{\boldmath$#1$\unboldmath}}

\def\vcsmall#1{\mbox{\boldmath$\scriptstyle #1$\unboldmath}}

\def\vczero{{\mathbf 0}}


%\def\beginlist{\begin{itemize}}
%
%\def\endlist{\end{itemize}}


\def\pr(#1){\left({#1}\right)}
\def\br[#1]{\left[{#1}\right]}
\def\fbr[#1]{\!\left[{#1}\right]}
\def\set#1{\left\{{#1}\right\}}
\def\ip<#1>{\left\langle{#1}\right\rangle}
\def\iip<#1>{\left\langle\!\langle{#1}\right\rangle\!\rangle}

\def\norm#1{\left\| #1 \right\|}

\def\abs#1{\left|{#1}\right|}
\def\fpr(#1){\!\pr({#1})}

\def\Re{{\rm Re}\,}
\def\Im{{\rm Im}\,}

\def\floor#1{\left\lfloor#1\right\rfloor}
\def\ceil#1{\left\lceil#1\right\rceil}


\def\mapengine#1,#2.{\mapfunction{#1}\ifx\void#2\else\mapengine #2.\fi }

\def\map[#1]{\mapengine #1,\void.}

\def\mapenginesep_#1#2,#3.{\mapfunction{#2}\ifx\void#3\else#1\mapengine #3.\fi }

\def\mapsep_#1[#2]{\mapenginesep_{#1}#2,\void.}


\def\vcbr{\br}


\def\bvect[#1,#2]{
{
\def\dots{\cdots}
\def\mapfunction##1{\ | \  ##1}
\begin{pmatrix}
		 \,#1\map[#2]\,
\end{pmatrix}
}
}

\def\vect[#1]{
{\def\dots{\ldots}
	\vcbr[{#1}]
}}

\def\vectt[#1]{
{\def\dots{\ldots}
	\vect[{#1}]^{\top}
}}

\def\Vectt[#1]{
{
\def\mapfunction##1{##1 \cr} 
\def\dots{\vdots}
	\begin{bmatrix}
		\map[#1]
	\end{bmatrix}
}}



\def\thetaB{\mbox{\boldmath$\theta$}}
\def\zetaB{\mbox{\boldmath$\zeta$}}


\def\newterm#1{{\it #1}\index{#1}}


\def\TT{{\mathbb T}}
\def\C{{\mathbb C}}
\def\R{{\mathbb R}}
\def\II{{\mathbb I}}
\def\F{{\mathcal F}}
\def\E{{\rm e}}
\def\I{{\rm i}}
\def\D{{\rm d}}
\def\dx{\D x}
\def\ds{\D s}
\def\dt{\D t}
\def\CC{{\cal C}}
\def\DD{{\cal D}}
\def\U{{\mathbb U}}
\def\A{{\cal A}}
\def\K{{\cal K}}
\def\DTU{{\cal D}_{{\rm T} \rightarrow {\rm U}}}
\def\LL{{\cal L}}
\def\B{{\cal B}}
\def\T{{\cal T}}
\def\W{{\cal W}}


\def\tF_#1{{\tt F}_{#1}}
\def\Fm{\tF_m}
\def\Fab{\tF_{\alpha,\beta}}
\def\FC{\T}
\def\FCpmz{\FC^{\pm {\rm z}}}
\def\FCz{\FC^{\rm z}}

\def\tFC_#1{{\tt T}_{#1}}
\def\FCn{\tFC_n}

\def\rmz{{\rm z}}

\def\chapref#1{Chapter~\ref{Chapter:#1}}
\def\secref#1{Section~\ref{Section:#1}}
\def\exref#1{Exercise~\ref{Exercise:#1}}
\def\lmref#1{Lemma~\ref{Lemma:#1}}
\def\propref#1{Proposition~\ref{Proposition:#1}}
\def\warnref#1{Warning~\ref{Warning:#1}}
\def\thref#1{Theorem~\ref{Theorem:#1}}
\def\defref#1{Definition~\ref{Definition:#1}}
\def\probref#1{Problem~\ref{Problem:#1}}
\def\corref#1{Corollary~\ref{Corollary:#1}}

\def\sgn{{\rm sgn}\,}
\def\Ai{{\rm Ai}\,}
\def\Bi{{\rm Bi}\,}
\def\wind{{\rm wind}\,}
\def\erf{{\rm erf}\,}
\def\erfc{{\rm erfc}\,}
\def\qqquad{\qquad\quad}
\def\qqqquad{\qquad\qquad}


\def\spand{\hbox{ and }}
\def\spodd{\hbox{ odd}}
\def\speven{\hbox{ even}}
\def\qand{\quad\hbox{and}\quad}
\def\qqand{\qquad\hbox{and}\qquad}
\def\qfor{\quad\hbox{for}\quad}
\def\qqfor{\qquad\hbox{for}\qquad}
\def\qas{\quad\hbox{as}\quad}
\def\qqas{\qquad\hbox{as}\qquad}
\def\qor{\quad\hbox{or}\quad}
\def\qqor{\qquad\hbox{or}\qquad}
\def\qqwhere{\qquad\hbox{where}\qquad}



%%% Words

\def\naive{na\"\i ve\xspace}
\def\Jmap{Joukowsky map\xspace}
\def\Mobius{M\"obius\xspace}
\def\Holder{H\"older\xspace}
\def\Mathematica{{\sc Mathematica}\xspace}
\def\apriori{apriori\xspace}
\def\WHf{Weiner--Hopf factorization\xspace}
\def\WHfs{Weiner--Hopf factorizations\xspace}

\def\Jup{J_\uparrow^{-1}}
\def\Jdown{J_\downarrow^{-1}}
\def\Jin{J_+^{-1}}
\def\Jout{J_-^{-1}}



\def\bD{\D\!\!\!^-}




\def\questionequals{= \!\!\!\!\!\!{\scriptstyle ? \atop }\,\,\,}

\def\elll#1{\ell^{\lambda,#1}}
\def\elllp{\ell^{\lambda,p}}
\def\elllRp{\ell^{(\lambda,R),p}}


\def\elllRpz_#1{\ell_{#1{\rm z}}^{(\lambda,R),p}}


\def\sopmatrix#1{\begin{pmatrix}#1\end{pmatrix}}


\def\bbR{{\mathbb R}}
\def\bbC{{\mathbb C}}


\begin{document}



\textbf{Numerical Analysis MATH50003 (2023\ensuremath{\endash}24) Problem Sheet 1}

\rule{\textwidth}{1pt}
\textbf{Problem 1} Assuming $f$ is differentiable, prove the left-point Rectangular rule error formula
\[
\ensuremath{\int}_a^b f(x) {\rm d}x =  h \ensuremath{\sum}_{j=0}^{n-1} f(x_j) +  \ensuremath{\delta}
\]
where $|\ensuremath{\delta}| \ensuremath{\leq} M (b-a) h$ for $M = \sup_{a \ensuremath{\leq} x \ensuremath{\leq} b}|f'(x)|$, $h = (b-a)/n$ and $x_j = a + jh$.

\textbf{SOLUTION}

This proof is very similar to the right-point rule, the only difference is we use a different constant in the indefinite integration in the integration-by-parts. First we need to adapt \textbf{Lemma 1 (Rect. rule error on one panel)}:
\meeq{
\ensuremath{\int}_a^b f(x) {\rm d}x = \ensuremath{\int}_a^b (x-b)' f(x)  {\rm d}x = [(x-b) f(x)]_a^b - \ensuremath{\int}_a^b (x-b) f'(x) {\rm d} x \ccr
= (b-a) f(a) + \underbrace{\left(-\ensuremath{\int}_a^b (x-b) f'(x) {\rm d} x \right)}_\ensuremath{\varepsilon}.
}
where
\[
\abs{\ensuremath{\varepsilon}} \ensuremath{\leq} (b-a) \sup_{a \ensuremath{\leq} x \ensuremath{\leq} b}|(x-b) f'(x)| \ensuremath{\leq} M (b-a)^2
\]
Applying this result on $[x_{j-1},x_j]$ we get
\[
\ensuremath{\int}_{x_{j-1}}^{x_j} f(x) {\rm d}x = h f(x_{j-1}) + \ensuremath{\delta}_j
\]
where $|\ensuremath{\delta}_j| \ensuremath{\leq} M h^2$. Splitting the integral into a sum of smaller integrals:
\[
\ensuremath{\int}_a^b f(x) {\rm d}x = \ensuremath{\sum}_{j=1}^n  \ensuremath{\int}_{x_{j-1}}^{x_j} f(x) {\rm d}x =
h \ensuremath{\sum}_{j=1}^n f(x_{j-1}) +  \underbrace{\ensuremath{\sum}_{j=1}^n \ensuremath{\delta}_j}_\ensuremath{\delta}
\]
where using the triangular inequality we have
\[
|\ensuremath{\delta}| = \abs{ \ensuremath{\sum}_{j=1}^n \ensuremath{\delta}_j} \ensuremath{\leq} \ensuremath{\sum}_{j=1}^n |\ensuremath{\delta}_j| \ensuremath{\leq} M n h^2 = M(b-a)h.
\]
\textbf{END}

\textbf{Problem 2(a)}  Assuming $f$ is twice-differentiable, prove a one-panel Trapezium rule error bound:
\[
\ensuremath{\int}_a^b f(x) {\rm d}x = (b-a) {f(a) + f(b) \over 2} +  \ensuremath{\delta}
\]
where $|\ensuremath{\delta}| \ensuremath{\leq} M (b-a)^3$ for $M = \sup_{a \ensuremath{\leq} x \ensuremath{\leq} b}|f''(x)|$.

\emph{Hint}: Recall from the notes
\[
\ensuremath{\int}_a^b {(b-x) f(a) + (x-a) f(b) \over b-a} \dx = (b-a) {f(a) + f(b) \over 2}
\]
and you may need to use Taylor's theorem. Note that the bound is not sharp and so you may arrive at something sharper like $|\ensuremath{\delta}| \ensuremath{\leq} 3(b-a)^3 M/4$. The sharpest bound is $|\ensuremath{\delta}| \ensuremath{\leq} (b-a)^3 M/12$ but that would be a significantly harder challenge to show!

\textbf{SOLUTION}

Recall from the notes:
\[
\ensuremath{\int}_a^b {(b-x) f(a) + (x-a) f(b) \over b-a} \dx = (b-a) {f(a) + f(b) \over 2}
\]
Thus we can find by integration by parts twice (noting that the integrand vanishes at $a$ and $b$):
\meeq{
\ensuremath{\delta} = \ensuremath{\int}_a^b \br[f(x) - {(b-x) f(a) + (x-a) f(b) \over b-a}] {\rm d}x \ccr
 = -\ensuremath{\int}_a^b (x-b) \br[f'(x) - {f(b)-f(a) \over b-a}] {\rm d}x \ccr
 = {(b-a)^2 \over 2} \br[f'(a) - {f(b)-f(a) \over b-a}] + \ensuremath{\int}_a^b {(x-b)^2 \over 2} f''(x) {\rm d}x
}
Applying \textbf{Proposition 1} we know
\[
\abs{f'(a) - {f(b)-f(a) \over b-a}} \ensuremath{\leq} M (b-a)/2
\]
Further we have
\[
\abs{\ensuremath{\int}_a^b {(x-b)^2 \over 2} f''(x) {\rm d}x } \ensuremath{\leq} {(b-a)^3 \over 2} M
\]
Thus we have the bound
\[
|\ensuremath{\delta}| \ensuremath{\leq} {(b-a)^2 \over 2} M (b-a)/2 + {(b-a)^3 \over 2} M \ensuremath{\leq} {3 (b-a)^3 \over 4} M \ensuremath{\leq} (b-a)^3 M.
\]
For the sharper $1/12$ constant check out the \href{https://en.wikipedia.org/wiki/Euler\ensuremath{\endash}Maclaurin_formula}{Euler\ensuremath{\endash}Maclaurin formula}.

\textbf{END}

\textbf{Problem 2(b)} Assuming $f$ is twice-differentiable, prove a bound for the Trapezium rule error:
\[
\ensuremath{\int}_a^b f(x) {\rm d}x = h \br[{f(a) \over 2} + \ensuremath{\sum}_{j=1}^{n-1} f(x_j) + {f(b) \over 2}] +  \ensuremath{\delta}
\]
where $|\ensuremath{\delta}| \ensuremath{\leq} M (b-a) h^2$ for $M = \sup_{a \ensuremath{\leq} x \ensuremath{\leq} b}|f''(x)|$.

\textbf{SOLUTION}

This is very similar to the rectangular rules: applying the preceding result on $[x_{j-1},x_j]$ we get
\[
\ensuremath{\int}_{x_{j-1}}^{x_j} f(x) {\rm d}x = h {f(x_{j-1}) + f(x_j) \over 2} + \ensuremath{\delta}_j
\]
where $|\ensuremath{\delta}_j| \ensuremath{\leq} M h^3$. Splitting the integral into a sum of smaller integrals:
\[
\ensuremath{\int}_a^b f(x) {\rm d}x = \ensuremath{\sum}_{j=1}^n  \ensuremath{\int}_{x_{j-1}}^{x_j} f(x) {\rm d}x =
h \br[{f(a) \over 2} + \ensuremath{\sum}_{j=1}^{n-1} f(x_j) + {f(b) \over 2}] +  \underbrace{\ensuremath{\sum}_{j=1}^n \ensuremath{\delta}_j}_\ensuremath{\delta}
\]
where using the triangular inequality we have
\[
|\ensuremath{\delta}| = \abs{ \ensuremath{\sum}_{j=1}^n \ensuremath{\delta}_j} \ensuremath{\leq} \ensuremath{\sum}_{j=1}^n |\ensuremath{\delta}_j| \ensuremath{\leq} M n h^3 = M(b-a) h^2.
\]
\textbf{END}

\rule{\textwidth}{1pt}
\textbf{Problem 3} Assuming $f$ is twice-differentiable, for the left difference approximation
\[
f'(x) = {f(x) - f(x - h) \over h} + \ensuremath{\delta},
\]
show that $|\ensuremath{\delta}| \ensuremath{\leq} Mh/2$ for $M = \sup_{x-h \ensuremath{\leq} t \ensuremath{\leq} x}\abs{f''(t)}$.

\textbf{SOLUTION}

Almost identical to the right-difference. Use Taylor series to right:
\[
f(x-h) = f(x) + f'(x) (-h) + {f''(t) \over 2} h^2
\]
where $t \ensuremath{\in} [x-h,x]$, so that
\[
f'(x) = {f(x) - f(x-h) \over h} + \underbrace{f''(t)/2 h}_\ensuremath{\delta}
\]
The bound follows immediately:
\[
|\ensuremath{\delta}| \ensuremath{\leq} |f''(t)/2 h| \ensuremath{\leq} Mh/2.
\]
\textbf{END}

\textbf{Problem 4} Assuming $f$ is thrice-differentiable, for the central differences approximation
\[
f'(x) = {f(x + h) - f(x - h) \over 2h} + \ensuremath{\delta},
\]
show that $|\ensuremath{\delta}| \ensuremath{\leq} Mh^2/6$ for $M = \sup_{x-h \ensuremath{\leq} t \ensuremath{\leq} x+h}\abs{f'''(t)}$.

\textbf{SOLUTION}

By Taylor's theorem, the approximation around $x+h$ is
\[
f(x+h) = f(x) + f'(x)h + \frac{f''(x)}{2}h^2 + \frac{f'''(t_1)}{6}h^3,
\]
for some $t_1 \ensuremath{\in} (x, x+h)$ and similarly $f(x-h) = f(x) + f'(x)(-h) + \frac{f''(x)}{2}h^2 - \frac{f'''(t_2)}{6}h^3,$ for some $t_2 \ensuremath{\in} (x-h, x)$.

Subtracting the second expression from the first we obtain $f(x+h)-f(x-h) = f'(x)(2h) + \frac{f'''(t_1)+f'''(t_2)}{6}h^3.$ Hence,
\[
\frac{f(x+h)-f(x-h)}{2h} = f'(x)  + \underbrace{\frac{f'''(t_1)+f'''(t_2)}{12}h^2}_{\ensuremath{\delta}}.
\]
Thus, the error can be bounded by $\left|\ensuremath{\delta}\right| \ensuremath{\leq} {M \over 6} h^2.$

\textbf{END}

\textbf{Problem 5}  Assuming $f$ is thrice-differentiable, for the second-order derivative approximation
\[
{f(x+h) - 2f(x) + f(x-h) \over h^2} = f''(x) + \ensuremath{\delta}
\]
show that $|\ensuremath{\delta}| \ensuremath{\leq} Mh/3$ for $M = \sup_{x-h \ensuremath{\leq} t \ensuremath{\leq} x+h}\abs{f'''(t)}$.

\textbf{SOLUTION} Using the same two formulas as in the previous problem we have $f(x+h) = f(x) + f'(x)h + \frac{f''(x)}{2}h^2 + \frac{f'''(t_1)}{6}h^3,$ for some $t_1 \ensuremath{\in} (x, x+h)$ and $f(x-h) = f(x) + f'(x)(-h) + \frac{f''(x)}{2}h^2 - \frac{f'''(t_2)}{6}h^3,$ for some $t_2 \ensuremath{\in} (x-h, x)$.

Summing the two we obtain $f(x+h) + f(x-h) = 2f(x) + f''(x)h^2 + \frac{f'''(t_1)}{6}h^3 - \frac{f'''(t_2)}{6}h^3.$

Thus, $f''(x) = \frac{f(x+h) - 2f(x) + f(x-h)}{h^2} + \frac{f'''(t_2) - f'''(t_1)}{6}h.$

Hence, the error is
\[
|\ensuremath{\delta}| = \left|f''(x) - {f(x+h) - 2f(x) + f(x-h) \over h^2} \right| = \left|\frac{f'''(t_2) - f'''(t_1)}{6}h\right|\ensuremath{\leq} {Mh \over 3}.
\]
\textbf{END}



\end{document}